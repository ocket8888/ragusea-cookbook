\documentclass{book}

\usepackage{xcookybooky}
\usepackage{units}
\usepackage{hyperref}
\usepackage{cookingsymbols}

\title{The Unofficial Adam Ragusea Cookbook}
\author{ocket8888 \and Adam Ragusea}

\begin{document}
\maketitle{}

\chapter*{Foreword}
Adam Ragusea is one of the authors listed on the front of this book, but you'll
notice the title is "The \emph{Unofficial} Adam Ragusea Cookbook". That's
because he wrote no part of it, is not affiliated with the effort, and most
likely is not even aware of its existence. This book is merely an attempt at
compiling his recipes and advice into a single publication - and a shoddy effort
at that. So while this book is dedicated to this author's favorite YouTube cook,
know that it is in no way associated with the man himself.\\
Some parts of the book may contain direct quotes of Adam Ragusea, while others
merely paraphrase and still others belong to this author alone, trying to convey
his ideas. No distinction is made between these situations, because that would
be a massive pain. So while it's possible some parts of the book are directly
copying him, it is never safe to assume any given collection of words are his,
and this book is in no way meant to assert any non-culinary opinion or stance
held by Adam.\\
Because this work is wholly unauthorized, it is almost certainly also incomplete
at any given point in time. Pull requests are welcome.\\
Redistribution of this publication is subject to the terms of its software
license, found in the LICENSE file.\\

\tableofcontents{}

\chapter*{Preface}
When reading recipes in this book, you'll note that (hopefully) all of them
contain measurements and temperatures in both US and metric units. Most of
Adam's viewers are in the US, and so those are mostly the measurements he uses.
In many cases, the measurements that are "canonical" are those US measurements,
and the metric ones have been calculated from them. This is where things get a
little fuzzy.\\
Adam prefers to measure things as little as possible, and generally doesn't like
to be constantly referring to a recipe as he cooks. His philosophy is:

\begin{quote}
I like for a recipe to get me in the ballpark, and then I like to eyeball and
improvise the rest.
\end{quote}

... and so to a certain degree the very idea of a recipe with strict
measurements is antithetical to the "Adam Ragusea Philosophy". This author does
not dispute this, but thinks that having measurements as a starting point is not
only helpful to newbies, but a must for those dieting (though this is not a
dieting cookbook). However, certain liberties have been taken in some unit
conversions. For example, the \ref{rec:chicken-pot-pie} recipe calls for
\unit[1]{lb}/\unit[450]{g} of carrots - but one pound is actually 453.5924
grams. This kind of rounding will be omnipresent, with the intention of getting
readers that use the metric system into the same "ballpark" as those using US
measurements. Always remember to mess with a recipe as much as you want; as Adam
says:

\begin{quote}
There are no rules.
\end{quote}

Finally, reading a recipe off a page can't really possibly impart as much
information as watching the video on which it's based, which is why every recipe
comes with a link to the original video, next to the "source" symbol. For the
best experience, it's recommended that you watch the video at least once and
then consult the written recipes as a short refresher when coming back to them.
For this reason, recipes generally leave out \emph{why} things are done as
that's better described by the videos, focusing instead on the raw mechanics.

\chapter{Recipes}

\begin{recipe}
[%
	preparationtime={\unit[1]{h}},
	bakingtime={\unit[20]{min}},
	bakingtemperature={\protect\bakingtemperature{fanoven=\unit[400]{$^\circ{}$F}/\unit[200]{$^\circ{}$C}}},
	portion={\portion{6-8}},
	source={\href{https://youtu.be/IbgpNZfi6Xo}{Chicken Pot Pie}}
]
{Chicken Pot Pie\label{rec:chicken-pot-pie}}
	\suggestion[Cauliflower]{
		Adam suggests you use purple cauliflower, if only because it's prettier.
		It does also contain anthocyanin antioxidants, which make it slightly
		healthier (and give it that purple color). Additionally, purple
		cauliflower is a bit sweeter and nuttier than its white cousin, which
		tends to go nicely with the savory and slightly nutty sauce.
	}

	\ingredients{
		\multicolumn{2}{l}{\textbf{Filling}}\\
		\hline
		Half & three or four pound rotisserie roasted chicken\\
		\unit[1]{lb}/\unit[450]{g} & Carrots\\
		1 & Cauliflower\\
		\unit[1]{Stick} & Butter\\
		\unit[1]{c}/\unit[250]{ml} & Flour\\
		\unit[$\frac{1}{2}$]{Bottle} & White Wine\\
		\unit[4]{c}/\unit[1]{L} & Milk\\
		\unit[1]{Tbsp} & Liquid Chicken Bouillon (optional)\\
		\unit[1]{Pinch} & Salt and Pepper\\
		\multicolumn{2}{l}{\textbf{Crust}}\\
		\hline
		\unit[2]{c}/\unit[500]{ml} & Flour\\
		\unit[$\frac{1}{2}$]{tsp}/\unit[2.5]{ml} & Salt\\
		\unit[1]{Stick} & Cold Butter\\
		\unit[4]{Tbsp}/\unit[60]{ml} & White Wine (cold)\\
		1 & Egg\\
		\unit[1]{Splash} & Water
	}
	\preparation{
		\begin{enumerate}
			\item Using a food processor, blend the flour, salt, and butter for
				the crust. Be sure to cut butter into chunks before putting it
				in.
			\item Blend in the white wine, one tablespoon at a time until dough
				can be formed into soft flakes with your fingers. You can use
				water instead, but white wine will probably taste better.
			\item Place dough onto plastic wrap and just \emph{barely} bring it
				together into a ball, working it as little as possible to
				accomplish that. Wrap the dough completely and leave in the
				refrigerator to chill.
			\item Peel the carrots - if you're the kind of person who doesn't
				like the skin of a carrot, it doesn't much matter, just a
				personal choice - and cut off the very tops and tips. Cut the
				carrots into even-length pieces.
			\item Chop the cauliflower florets into somewhat large, bite-sized
				pieces.
			\item Melt a stick of butter in a large saucepan.
			\item Whisk the flour for the filling into the melted butter without
				removing it from the heat.
			\item When the roux starts to smell a bit nutty, scoop some out and
				set it aside. The amount you scoop depends on how thick you want
				the sauce; in fact, the only reason we set any aside is so you
				can thicken the sauce later if it's too thin for your taste. It
				looks like Adam takes out a little more than
				\unit[$\frac{1}{4}$]{c}/\unit[50]{ml} - but then he adds it all
				back in later.
			\item Whisk in about half a bottle of white wine, a little at a
				time. If you're confused about what size of bottle is implied,
				know that half of whatever bottle Adam uses is about
				\unit[2]{c}/\unit[500]{ml}.
			\item Whisk in four cups of milk, a little at a time.
			\item Add in (or don't) the set-aside roux from earlier until the
				sauce has the desired thickness.
			\item Stir in a tablespoon of liquid chicken bouillon if desired.
			\item Add the chopped carrots, stirring in with a spatula now
				instead of a whisk.
			\item Preheat the oven if you haven't done so already.
			\item When the carrots are about half done, add in the cauliflower.
			\item Chop chicken into bite-sized pieces and add to the sauce.
			\item At this point, if the mixture seems too thick, feel free to
				splash in some extra milk to thin it out.
			\item When the cauliflower is about half cooked, sprinkle in salt
				and pepper to taste. The sauce should ultimately taste a little
				too salty, as it doesn't need to stand on its own.
			\item Flour a space for rolling out the dough - cutting board or
				counter top, whichever you prefer.
			\item Roll the dough out just a little with a pin, then turn it and
				roll again. If it happens to tear and form holes, simply rip a
				outer piece of dough off to graft over the hole and keep
				rolling. Don't worry if it looks like the dough is starting to
				fall apart. To know when you're done rolling, compare the size
				of the rolled-out dough to your
				\unit[2$\frac{1}{2}$-3]{qt}/\unit[2$\frac{1}{2}$-3]{L} pie dish.
				You want it to extend at least an inch beyond the edges of the
				top of the dish, so keep rolling until it does.
			\item Crack an egg into a bowl and whisk it smooth with some water.
				Using a pastry brush, smear the egg wash along the edges of the
				dish.
			\item Dump frozen peas into the sauce right before taking it off the
				heat.
			\item Pour the filling into the pie dish, gently draping the dough
				over the top of it and brush the top of the dough with some of
				the egg wash from earlier.
			\item Place the pie in the oven. Adam recommends you place some
				aluminum foil or something beneath the pie to catch any sauce
				that might boil over.
			\item Bake for 20 minutes or until crust is nicely browned.
			\item Wait at least 10 minutes for the pie to cool before serving it
				forth.
		\end{enumerate}
	}
\end{recipe}
\end{document}

